\section{Theoretical Analysis}
\label{sec:analysis}

To theoretically analyse the circuit, as stated before, we make use of two circuit analysing methods, the Meshes' Method and the Nodes' Method.

The Meshes' Method consists in writing Kirchhoff's Voltage Law (KVL) for every mesh in the circuit. To do so we assume that every mesh has a current associated with it, which flows only in this mesh. We call this currents mesh currents. It is with this mesh currents that we may write the KVL. Per example, for resistor $R1$ we write its voltage as $V_{R1} =R_1*I_{M1}$ where $I_{M1}$ is the mesh current that flows through the upper-left mesh. For circuit components in the border between two meshes the current flowing through them is the sum of the adjacent mesh currents. Per example,assuming that all mesh currents flow clockwise, for resistor $R3$, the current flowing through it is given by $I_{R3} = I_{M1}-I_{M2}$. We then write voltage between the resistor's terminals using Ohm's Law the same way we did with resistor $R1$.

When analysing a circuit with the Nodes' Method we associate different voltages to every node, as we associated different currents to every mesh. By writing Kirchhoff's Current Law (KCL) for every node that is not connected to a voltage source we obtain the equations which will determine the voltage values in each node. Because most of the times this equations are not enough to all nodes' voltage, we may obtain other equations through direct circuit analysis. Per example, defining the node that connects $V_a$, $R_4$ and $R_6$ as the $0V$ node, we have that the voltage in the upper-right node, which we may call node A is equal to the voltage created by voltage source $V_a$. That is, $V_A = V_a$. KCL must be written in terms of the nodes' voltages. To do so we only need to apply Ohm's Law to obtain the current flowing through the resistors in terms of its terminal's voltage.

\section{Time response}

The circuit consists of a single V-R-C loop where a current $i(t)$ circulates. The
voltage source $v_I(t)$ drives its input, and the output voltage $v_O(t)$ is taken from
the capacitor terminals. Applying the Kirchhoff Voltage Law (KVL), a single
equation for the single loop in the circuit can be written as

\begin{equation}
  Ri(t) + v_O(t) = v_I(t).
  \label{eq:kvl}
\end{equation}

Because $v_O$ is the voltage between capacitor C's plates, it is related to the
current $i$ by
\begin{equation}
  i(t) = C\frac{dv_O}{dt}.
\end{equation}

Hence, Equation~(\ref{eq:kvl}) can be rewritten as
\begin{equation}
  RC\frac{dv_O}{dt} + v_O(t) = v_I.
  \label{eq:kvl2}
\end{equation}

Equation~(\ref{eq:kvl2}) is a linear differencial equation whose solution is a
superposition of a natural solution $v_{On}$ and a forced solution $v_{Of}$:

\begin{equation}
  v_O(t) = v_{On}(t) + v_{Of}(t).
  \label{eq:vo_sol}
\end{equation}

As learned in the theory classes the natural solution is of the form
\begin{equation}
  v_{On}(t) = Ae^{-\frac{t}{RC}},
  \label{eq:vo_nat}
\end{equation}
where $A$ is an integration constant.

The forced solution is of the form given in Equation~(\ref{eq:vo_for}) and is
illustrated in Figure~\ref{fig:forced}.

\begin{equation}
  V_{Of}(t) = |\bar{V}_{Of}| cos(\omega t + \angle \bar{V}_{Of}),
  \label{eq:vo_for}
\end{equation}

\lipsum[1-1]


\begin{figure}[h] \centering
\includegraphics[width=0.8\linewidth]{forced.eps}
\caption{Forced sinusoidal response.}
\label{fig:forced}
\end{figure}

\section{Frequency response}

\lipsum[1-1]


