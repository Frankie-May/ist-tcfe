
\section{Theoretical Analysis}
\label{sec:analysis}

To theoretically analyze this Bandpass filter circuit the following steps were made:
\par
\par
It is known that with the circuit connected as stated previously, (with a low and high pass filter and a gain stage created by the Operational Amplifier OP-AMP) the transfer function is the following:
\begin{equation}
    \dfrac{V_0(s)}{V_I(s)} = \dfrac{R_{lcut} C_{lcut} s}{1+R_{lcut} C_{lcut} s}\times \left( 1+\dfrac{R_2}{R_1} \right) \times \dfrac{1}{1+R_{hcut} C_{hcut} s}
\end{equation}
This makes it possible to plot the gain of the circuit with respect to the frequency. It is also possible to represent the phase with respect to the same variable.
\par
\par

We calculated the low cut-off frequency and the high cut-off frequency that is given by the following expressions:
\begin{equation}
    \omega_{low} = \dfrac{1}{R_{lcut} C_{lcut}}
\end{equation}

\begin{equation}
    \omega_{high} = \dfrac{1}{R_{hcut} C_{hcut}}
\end{equation}
Which then allowed to calculate the central frequency thas is calculated as follows:
\begin{equation}
    \omega_{0} = \sqrt{\omega_{low} \omega_{high}}
\end{equation}

We also calculated the output voltage for that frequency and plotted the input and output voltage to allow comparison.

Through octave, we were able to predict some simulation results. Because with octave we used OP-AMP (Operational Amplifier)  models, the values and plots obtained through mathematical analysis differ from those obtained in the $\ref{sec:simulation}^{rd}$ section.

The results are as follow:

\begin{figure}[h] 
\centering
\includegraphics[width=0.6\linewidth]{gain.eps}
\caption{Theoretical Gain in dB.}
\label{Fig2: TheoGaindB}
\end{figure}

\begin{table}[H]
\centering
\begin{tabular}{|l|l|}
\hline
{\bf Name} & {\bf Value} \\ \hline
    \input{gainstg}
\end{tabular}
\caption{\textbf{Gain Stage Theoretical Gain, Output Impedance and Input Impedance.}}
\end{table}

\begin{table}[H]
\centering
\begin{tabular}{|l|l|}
\hline
{\bf Name} & {\bf Value} \\ \hline
    \input{outputstg}
\end{tabular}
\caption{\textbf{Output Stage Theoretical Gain, Output Impedance and Input Impedance.}}
\end{table}

\begin{table}[H]
\centering
\begin{tabular}{|l|l|}
\hline
{\bf Name} & {\bf Value} \\ \hline
    \input{total}
\end{tabular}
\caption{\textbf{Circuit Theoretical Gain, Output Impedance and Input Impedance.}}
\end{table}
