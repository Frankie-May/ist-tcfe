\section{Conclusion}
\label{sec:conclusion}

The Work's Cost and Merit is given in the following table:

\begin{table}[H]
\centering
\begin{tabular}{|l|l|}
\hline
{\bf Name} & {\bf Value} \\ \hline
    \input{MERIT_TAB}
\end{tabular}
\caption{Cost and Merit}
\end{table}

With the obtained results, somewhat similar either in the simulation and the theoretical analysis, we can say that the objective of the work was accomplished and we were able to amplify an input signal wave and select a specified frequency through a circuit composed by an OP-AMP (Operational Amplifier), resistances and capacitors. The obtained merit is not very big, but it was not possible to reduce it much more due to the high OPAMP cost and the limited number of resistors and capacitors we could have used. Even so, the merit is pretty reasonable.

PS: We added a .txt file in the top folder in which there are values and instructions on how to change some variable values in order to obtain a better merit and more similar results in the simulation and in the theoretical analysis. This is not automatized and the values need to be changed by hand.