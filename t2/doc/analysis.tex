\section{Theoretical Analysis}
\label{sec:analysis}
The theoretical analysis is divided in 6 different points.
In the first point we must analyze the circuit when a direct current is imposed in the circuit by the $Vs(t)$ source. This value was given in the data file and is refered to as Vs in the subsequent writing. Because there is no alternate current acting, the circuit is opened in the capacitor's branch. This circuit is then analyzed with resource to Nodal Analysis. The following Matrix Equation was used to determine voltage in every node of the circuit and the currents in each branch were then calculated thorugh Ohm's Law.

\begin{equation}
\begin{bsmallmatrix}
1  & 0  & 0  & 0  & 0  & 0  & 0\\
G1  & -(G1 + G2 + G3)  & G2  & G3  & 0  & 0  & 0\\
0  & -(G2 + Kb)  & G2  & Kb  & 0  & 0  & 0\\
0  & -Kb  & 0  & (G5 + Kb)  & -G5  & 0  & 0\\
0  & 0  & 0  & 0  & 0  & (G6 + G7)  & -G7\\
0  & 0  & 0  & 1  & 0  & (Kd * G6)  & -1\\
0  & G3  & 0  & (-G4 - G3 - G5)  & G5  & G7  & -G7
\end{bsmallmatrix}
*
\begin{bsmallmatrix}
V1\\
V2\\
V3\\
V5\\
V6\\
V7\\
V8
\end{bsmallmatrix}
=
\begin{bsmallmatrix}
Vs\\
0\\
0\\
0\\
0\\
0\\
0
\end{bsmallmatrix}
\end{equation}

\vspace{20pt}

The second analysis point consists in the determination of the equivalent resistance ($R_{eq}$) of the circuit when seen from the capacitor's branch. This is done with resource to two substitutions. Firstly we trade $V_s(t)$ voltage source for a 0V source, creating a short circuit between nodes GND and 1. Then we trade the capacitor for a voltage source whose voltage given ($V_x$) is equal to the voltage between nodes $V_6$ and $V_8$ determined in the previous step. $V_x$ is therefore determined with the following equation:
\begin{equation}
V_x=V_6-V_8
\end{equation}
After this we the determine the currrent ($I_x$) flowing through the $V_x$ voltage source (the voltage source which we traded the capacitor with) and find the $R_{eq}$ through the following equation:
\begin{equation}
R_eq=V_x/I_x
\end{equation}

The following matricial equation corresponds to the application of the modified nodes method to the circuit in which we write equations for every node and every relation between voltage sources and voltages in nodes. Vs equals zero to create a short circuit between nodes GND and 1.

\begin{equation}
\begin{bsmallmatrix}
1 & 0 & 0 & 0 & 0 & 0 & 0 & 0 & 0\\
G1 & -(G1 + G2 + G3) & G2 & G3 & 0 & 0 & 0 & 0 & 0\\
0 & -(G2+Kb) & G2 & Kb & 0 & 0 & 0 & 0 & 0\\
0 & 0 & 0 & 0 & 0 & (G6 + G7) & -G7 & 0 & 0\\
0 & 0 & 0 & 0 & 1 & 0 & -1 & 0 & 0\\
0 & 0 & 0 & 1 & 0 & (Kd * G6) & -1 & 0 & 0\\
0 & Kb & 0 & -(G5 + Kb) & G5 & 0 & 0 & 1 & 0\\
0 & G3 & 0 & (-G4-G3-G5) & G6 & 0 & 0 & 0 & -1\\
0 & 0 & 0 & 0 & 0 & -G7 & G7 & -1 & -1
\end{bsmallmatrix}
*
\begin{bsmallmatrix}
V1\\V2\\V3\\V5\\V6\\V7\\V8\\Ix\\Iy
\end{bsmallmatrix}
=
\begin{bsmallmatrix}
Vs\\0\\0\\0\\Vx\\0\\0\\0\\0
\end{bsmallmatrix}
\end{equation}

\vspace{20pt}
This is a needed procedure in order to determine the time constant for the circuit, $\tau$. The following expression is valid only for the time constant of a RC constant
\begin{equation}
\tau=R*C
\end{equation}
 With this constant and knowing that the circuit will discharge the capacitor the general solution for the circuit will be given by the following expression:
\begin{equation}
V(t)=V(0)*e^(-(t/\tau))
\end{equation}


The normal solution for node 6 is plotted in the third part of this analysis, using the values for $V_6(0)$ obtained in the previous step and the time constant obtained using the capacitor's $R_{eq}$. This is plotted in an interval of $[0, 20]ms$.
\begin{figure}[h] \centering
\includegraphics[width=0.6\linewidth]{graf_nat.pdf}
\caption{Plot for $V_6(t)$ natural solution}
\end{figure}

Fourth step consists in determining the forced solution for node 6 in the same interval ($[0, 20]ms$). To do this we once again replace the Voltage source $V_s$ by a voltage source with amplitude 1, but in this case with an argument of $-\pi/2$ given that we are representing $V_s$ in its complex form in order to make calculations faster. The capacitor is replaced by its impedance which is given by the following equation:
\begin{equation}
Z_c=1/(j*\omega*C)
\end{equation}
Where $Z_c$ is the capacitors impedance, j is the imaginary unit, $\omega$ is the angular speed given by $\omega=2*\pi*f$ and C is the capacitor's capacity which was given in the data. Frequency f in this step is equal to $f=1kHz$.
Then we run nodal analysis in the circuit in order to find the complex amplitudes for each node. The following matrix describes the equations used to perform this. The equations used are given by the application of the regular nodal analysis, using equations from nodes not connected to voltage source, the $V_d$ voltage relationship with nodes $V_8$ and $V_5$ and the super-node equation from the 5-8 super-node.
\begin{equation}
\begin{bsmallmatrix}
1 & 0 & 0 & 0 & 0 & 0 & 0\\
1/Z1 & -(1/Z1 + 1/Z2 + 1/Z3) & 1/Z2 & 1/Z3 & 0 & 0 & 0\\
0 & -(1/Z2 + Kb) & 1/Z2 & Kb & 0 & 0 & 0\\
0 & -Kb & 0 & (1/Z5 + Kb) & -(1/Z5+1/Zc) & 0 & 1/Zc\\
0 & 0 & 0 & 0 & 0 & (1/Z6 + 1/Z7) & -1/Z7\\
0 & 0 & 0 & 1 & 0 & (Kd * 1/Z6) & -1\\
0 & 1/Z3 & 0 & -(1/Z4 + 1/Z3 + 1/Z5) & 1/Z5+1/Zc & 1/Z7 & -(1/Z7+1/Zc)\\
\end{bsmallmatrix}
*
\begin{bsmallmatrix}
V1\\V2\\V3\\V5\\V6\\V7\\V8
\end{bsmallmatrix}
=
\begin{bsmallmatrix}
Vsf\\0\\0\\0\\0\\0\\0
\end{bsmallmatrix}
\end{equation}
\par
\par
\begin{table}
\centering
\caption{Complex Amplitudes for all Nodes}
\begin{tabular}{|c|c|}
\hline
 {\bf Voltage} & {\bf Value [V]}
 \input{tcamp.tex}
\end{table}
Fifth step consists in using the amplitudes determined in the previous step to write real time functions for nodes. This is done by sum of the forced solutions with the natural solutions, assuming a frequency $f=1kHz$. Both $V_s(t)$ and $V_6(t)$ are plotted in the $[-5, 20]ms$ interval.
\begin{figure}[h] \centering
\includegraphics[width=0.6\linewidth]{graf_V6_t.pdf}
\caption{Plot for $V_s(t)$ and $V_6(t)$ total solutions}
\end{figure}

In the final step, frequency responses are plotted for node $V_6$, for voltage source $V_s$ and for the capacitor's voltage $V_c=V_6+V_8$. 
\begin{figure}[h] \centering
\includegraphics[width=0.6\linewidth]{graf_mag.pdf}
\caption{Plot for magnitude frequency responde for $V_6$, $V_s$ and $V_c$}
\end{figure}
\begin{figure}[h] \centering
\includegraphics[width=0.6\linewidth]{graf_phase.pdf}
\caption{Plot for phase frequency responde for $V_6$, $V_s$ and $V_c$}
\end{figure}
Tables are input after both simulation and analysis section for best comparison.
